\documentclass[mathserif,9pt]{beamer}
\usepackage{multimedia}
%%\usepackage[utf8]{inputenc}
%%\usetheme{Warsaw}
%%\usetheme{Rochester}
%%\usecolortheme{whale}
\usepackage{hyperref}
\hypersetup{colorlinks=true,filecolor=red,pdfnewwindow=true,pdffitwindow=false}

\definecolor{oxygenorange}{HTML}{F7800A}
\definecolor{oxygengray}{HTML}{686868}
\definecolor{oxygenlightgray}{HTML}{EEEEEE}
\definecolor{oxygenblue}{HTML}{236EAF}
\definecolor{ox}{HTML}{111EAF}
\definecolor{my_purple}{HTML}{ca1885}
%%\definecolor{my_anis}{HTML}{b3ec51}
%%\definecolor{my_anis}{HTML}{c2ed3d}
\definecolor{my_anis}{HTML}{e3ed3d}
\definecolor{my_tilleul}{HTML}{9a9c16}
\definecolor{darkred}{HTML}{c21125}
\definecolor{my_green}{HTML}{88ac0c}

\definecolor{blue_bruz}{HTML}{235497}
\definecolor{b_bruz}{HTML}{207ede}
\definecolor{grey_bruz}{HTML}{50586d}
\definecolor{yellow_bruz}{HTML}{b7a170}

\setbeamercolor*{Title bar}{fg=white}


\def\H{{\mathcal H}}
\def\Hu{{\H^1}}
\def\Comp{{\;^{c}\,}}
\def\fint{{\int\hspace{-8mm}-}}

\def\a{{\alpha }}
\def\b{{\beta }}
\def\d{{\delta }}
\def\e{{\varepsilon}}
\def\o{{\omega}}
\def\ds{\displaystyle}

\def\ha{{\hat{a}}}
\def\hb{{\hat{b}}}
\def\hx{{\hat{x}}}
\def\vf{{\varphi}}
\def\Cau{{{\cal C}^{0,\alpha}(\Gamma_1)}}
\def\Cad{{{\cal C}^{0,\alpha}(\Gamma_2)}}
\def\R{\mathbf R}

%%%%%%%%%%%%%%%%%%%%%%%%%%%%%%%%%%%%%%%%%%%%%%%%%%%%%%%%%%%%%%%%%%%%%%%%%%%%%%%%%  


\begin{document}
\usebackgroundtemplate{}

%%--------------------------------------------------------------------------------
\begin{frame}

\textcolor{b_bruz}{\bf{\Large{Homogenization and the Neumann Poincar\'e operator}}}
\vspace*{1cm}

\small{
Eric Bonnetier, Charles Dapogny, Faouzi Triki
\vspace*{10mm}

\textcolor{blue_bruz}{\bf Outline:}
\medskip

\begin{itemize}
\item[1.] Motivation : resonant frequencies in metallic
nanoparticles 

\item[2.]
The NP operator/the Poincar\'e variational problem
for a periodic collection of inclusions

\item[3.]
The limiting spectra

\item[4.]
Consequences concerning the homogenization of 
inclusions with non-positive conductivities

\item[5.]
Conclusion
\end{itemize}

}
\end{frame}
%%--------------------------------------------------------------------------------
\begin{frame}
\small{

\textcolor{b_bruz}{\Large{\bf 1. Resonant frequencies of metallic nanoparticles}}

\begin{figure}[hbt]
\includegraphics[angle=0,height=20mm]{../Transp/foiltex/nanoparticles_1.jpg}
\hspace*{4mm}
\includegraphics[angle=0,height=20mm]{Im_Presentation/vitrail_2}
\end{figure}
%%\textcolor{red}{[Gang Bi et al, Optics Comm., 285 (2012) 2472]}
\bigskip

Very small metallic particles exhibit interesting diffractive phenomena,
related to resonances~: \textcolor{b_bruz}{localization and extremely large 
enhancement of the electromagentic fields in their vicinity}
\medskip

Many potential applications~: \textcolor{black}{nanophotonics, 
nanolithography, near field microscopy,  biosensors, cancer therapy}
\medskip

\textcolor{b_bruz}{\bf 2 main ingredients :}
\begin{itemize}
\item[-]
The wavelength of the incident excitation should be larger than
the particle diameter

\item[-]
the real part of the electric permittivity $\varepsilon(\omega)$ 
inside the particle is negative 
\end{itemize}


}
\end{frame}
%------------------------------------------------------------
\begin{frame}
\small{
\textcolor{b_bruz}{\bf Typical model problem}
\medskip

Consider a particule that occupies a bounded  $C^2$ domain 
\textcolor{ox}{$\delta D \subset \R^d$}
\medskip 

$\delta$ is small, $|D| = 1$
\medskip

$\omega \in \mathbf{C}$ is a resonant frequency of the nanoparticle
$D_\delta$  if  there exists a  non-trivial solution $U$ to the PDE 
(TE polarization):
\medskip

\textcolor{ox}{
\[
\left\{ \begin{array}{llccc}
\Delta U +\omega^2 \varepsilon(x, \omega)\mu_0 U &= \;0
&\qquad \textrm{in}& \R^d \setminus \overline{D_\delta} \cup D_\delta
\\ 
\left[\varepsilon U\right] &=\; 0& 
\qquad \textrm{on}& \partial D_\delta
\\ 
\left[ \frac{\partial U}{\partial \nu}\right] &=\; 0& 
\qquad \textrm{on}& \partial D_\delta
\\
\textrm{radiation condition}
\end{array}\right.
\]}
\bigskip

The Drude model gives a good description of the electric permittivity $\varepsilon$
of metals such as Au, Ag, Al, in the range of frequencies of interest

\textcolor{ox}{
\begin{eqnarray*}
\varepsilon(x, \omega) &=& 
\left\{ \begin{array}{ll} 
\varepsilon_0 &\quad \textrm{for}\;x \in \R^d \setminus \overline{D_\delta}
\\
\varepsilon_0 \hat \epsilon(\omega) \;=\; 
\varepsilon_0\left(
\varepsilon_\infty - \frac{\omega_P^2}{\omega^2+i\omega \Gamma}
\right)
& \quad \textrm{for}\; x \in D_\delta
\end{array} \right.
\end{eqnarray*}}

%\end{eqnarray*}}
%where $\varepsilon_\infty>0,~\omega_P>0$ and~$\Gamma>0$ 
%\hat{\epsilon}(\omega) 


}
\end{frame}
%------------------------------------------------------------
\begin{frame}
\small{

The change of variable $\quad \tilde{x} = z + x/\delta \quad$ transforms the original
PDE into
\textcolor{ox}{
\[
\left\{ \begin{array}{llcl}
\Delta \tilde{U} + \d^2 \omega^2 \varepsilon(x, \omega)\mu_0 \tilde{U} &= \;0
&\qquad \textrm{in}\; \R^2\setminus \overline{D} \cup D
\\ [6pt]
\left[\varepsilon \tilde{U}\right] &=\; 0& 
\qquad \textrm{on}\; \partial D
\\ [6pt]
\left[ \frac{\partial \tilde{U}}{\partial \nu}\right] &=\; 0& 
\qquad \textrm{on}\; \partial D
\end{array}\right.
\]}
\medskip

where $\quad \tilde{U}(x) = U(\tilde{x}) \quad$ and one expects
that $\e \tilde{U}$ converges to a solution of the quasistatic problem
\textcolor{ox}{
\[
\left\{ \begin{array}{ccll}
\textrm{div}(1/\e(\omega) \nabla u) &=& 0 & \textrm{in}\; \R^d
\\
u& \to & 0 & \textrm{as}\;|x| \to \infty
\end{array}\right.
\]}
\medskip

\textcolor{ox}{
Electrostatic resonances: find the values of $\e$ for which the above PDE 
has non trivial solutions}
\medskip



\textcolor{red}{[Mayergoyz-Fredkin-Zhang, Grieser, Ammari-Millien-Ruiz-Zhang]} 



}
\end{frame}
%%--------------------------------------------------------------------------------
\begin{frame}
\small{


We may seek $u$ in the form 
\textcolor{ox}{$\quad u(x)\;=\; S_{D} \vf(x) \quad$}
where $S_D$ is the single layer potential on $\partial \Omega$

\textcolor{ox}{
\begin{eqnarray*}
S_D \psi(x) &=& \ds\int_{\partial D} G(x,y) \psi(y) \, d\sigma(y),
\quad x \in \R^d
\\
G(x,y) &=& \left\{ \begin{array}{ll}
\ds\frac{1}{2\pi} \ln|x-y| & \textrm{if}\; d=2
\\
\ds\frac{|x-y|^{d-2}}{(2-d)\omega_d} & \textrm{if}\; d \geq 3
\end{array} \right.
\end{eqnarray*}}
\medskip

For $\psi \in L^2(\partial D)$,
the function $S_D \psi$ is harmonic in $D$ and in $\R^d \setminus D$,
continuous across $\partial D$ and satisfies the Pelmelj jump relations
\textcolor{ox}{
\begin{eqnarray*}
\partial_\nu S_D \psi |_\pm
&=& \pm 1/2 \psi + K^*_{D} \psi
\end{eqnarray*}}
\medskip

The operator $K^*_{D}$ (or its adjoint) is the Neumann-Poincar\'e operator
\textcolor{red}{
\begin{eqnarray*}
K^*_D \psi(x)
&=& \ds\int_{\partial D} 
\ds\frac{\nu(x)\cdot(x-y)}{|x-y|^2} \psi(y) \,d\sigma(y)
\end{eqnarray*}}
 
}
\end{frame}
%%--------------------------------------------------------------------------------
\begin{frame}
\small{

The layer potential $\vf$ yields a solution to the PDE provided
\textcolor{ox}{
\begin{eqnarray*}
(\lambda(\omega) I - K^*_D)\vf &=& 0
\end{eqnarray*}}

where \textcolor{ox}{$\quad \lambda(\omega) \;=\; \ds\frac{1/\hat{\epsilon}(\omega) + 1}
{2(1/\hat{\epsilon}(\omega) - 1)}\quad$}
is thus an eigenvalue of $K^*_D$
\bigskip

\begin{itemize}
\item[-]
$\sigma(K^*_D) \subset [-1/2,1/2]$
\medskip

\item[-]
When $D$ is smooth (${\mathcal C}^{1,\a}$), 
$K^*_D$ is compact consisting of a countable sequence of eigenvalues accumulating
at $0$
\medskip

\item[-]
When $D$ is Lipschitz, $K^*_D$ may have continous spectrum
\end{itemize}
 
}
\end{frame}
%%--------------------------------------------------------------------------------
\begin{frame}
\small{


Goal in applications: tune the shape of $D$ to trigger 
resonant frequencies at desired values of $\omega$

\begin{figure}[hbt]
\includegraphics[angle=0,height=20mm]{../Transp/foiltex/nanoparticles_2.jpg}
\end{figure}
\textcolor{red}{[Gang Bi et al, Optics Comm., 285 (2012) 2472]}
\bigskip

The Neumann-Poincar\'e operator naturally appears also in other 
situations: cloaking, pointwise estimates on gradients of solutions to elliptic PDE's
in composite media
\medskip

\textcolor{red}{[Ammari-Ciraolo-Kang-Lim, Perfekt-Putinar, Ola, Kang-Lim-Yu, EB-Triki]}


}
\end{frame}
%------------------------------------------------------------
\begin{frame}
\small{


\textcolor{orange}{\Large{\bf 2. The Neumann-Poincar\'e operator/
Poincar\'e variationnal problem
for a periodic collection of inclusions}}

\begin{figure}[hbt]
\includegraphics[angle=0,width=45mm]{../Charles/NPoperator/setting}
\end{figure}

\vspace*{-14mm}
\hspace*{80mm}$\omega \subset\subset Y$
\vspace*{14mm}

Consider $\Omega \subset \R^2$, smooth bounded domain, that contains
a periodic collection of smooth inclusions 
\textcolor{ox}{
\begin{eqnarray*}
D \;=\; \omega_\e \;=\; \cup_{i \in N_\e} (\omega_{\e,i})
\quad\quad\quad
\omega_{\e,i} = z_{\e,i} + \e \omega, \quad i \in N_{\e,i}
\end{eqnarray*}}
\medskip

Model PDE : given $f \in L^2(\Omega)$, seek $u \in H^1_0(\Omega)$ such that
\textcolor{ox}{
\begin{eqnarray*}
-\textrm{div}(A(x) \nabla u) &=& f \quad \textrm{in}\; \Omega,
\quad\quad\quad
A(x) =
\left\{ \begin{array}{ll}
k & \textrm{in}\; \omega_\e
\\
1 & \textrm{otherwise}
\end{array} \right.
\end{eqnarray*}}


\textcolor{orange}{\bf What are the resonant frequencies of such a system ?
Are there collective effects ?}
\medskip

\textcolor{orange}{\bf What is $\lim_{\e \to 0} \sigma(K^*_\e)$ ?}


}
\end{frame}
%------------------------------------------------------------
\begin{frame}
\small{

As the definition of the Neumann-Poincar\'e depends on the number of inclusions,
we work with the Poincar\'e variational operator

\textcolor{ox}{
\[
T_{\e}~: H^1_0(\Omega) \rightarrow H^1_0(\Omega)
\]}
\vspace*{-5mm}

\textcolor{ox}{
\begin{eqnarray*}
\forall\; v \in H^1_0(\Omega), \quad
\int_\Omega \nabla T_\e u \cdot \nabla v
&=&
\int_{\omega_\e} \nabla u \cdot \nabla v
\end{eqnarray*}}
\bigskip

If $\quad T_\e u \;=\; \beta u \quad$ for some $u \in H^1_0(\Omega)$ and $\beta \in \R$,
then for any $v \in H^1_0(\Omega)$
\textcolor{ox}{
\begin{eqnarray*}
\lefteqn{
\ds\int_\Omega \nabla T_\e u \cdot \nabla v
\;-\; \ds\int_{\omega_\e} \nabla u \cdot \nabla v
\;=\; 
\beta \ds\int_\Omega \nabla u \cdot \nabla v
\;-\;\ds\int_{\omega_\e} \nabla u \cdot \nabla v}
\\
&=&
\beta \ds\int_{\Omega \setminus \omega_\e} \nabla u \cdot \nabla v
+ \left(\beta -1 \right) \int_{\omega_\e} \nabla u \cdot \nabla v
\;=\; 0
\end{eqnarray*}}
\medskip

It follows that \textcolor{ox}{$\quad \textrm{div}(a(\beta) \nabla u) = 0$}
\textcolor{ox}{
\begin{eqnarray*}
u = S_{\omega_\e}\vf &\textrm{with}&
(\lambda I - K^*_{\omega_\e}) \vf \;=\; 0,
\quad\quad\quad
\lambda = 1/2 - \beta
\end{eqnarray*}}


We conclude that 
\textcolor{b_bruz}{
$\quad \sigma(T_\e) = 1/2 - \sigma(K^*_{\omega_\e})$ }



}
\end{frame}
%------------------------------------------------------------
\begin{frame}
\small{

{\bf Theorem}
\textcolor{b_bruz}{
\begin{eqnarray*}
\lim_{\e \to 0} \sigma(T_\e) &=&
\{0,1\} \cup \sigma_{\textrm{Bloch}} \cup \sigma_{\partial \Omega}
\end{eqnarray*}}
\medskip

\textcolor{b_bruz}{$\bullet$}
The first term is the Bloch spectrum and corresponds to bulk
resonant modes of single cells or group of cells
\textcolor{b_bruz}{
\begin{eqnarray*}
\sigma_{\textrm{Bloch}} &=&
\cup_{i \geq 1}
[ \min_{\eta \in [0,1]^d} \lambda_i^-(\eta), 
\max_{\eta \in [0,1]^d} \lambda_i^-(\eta) ]
\cup
[ \min_{\eta \in [0,1]^d} \lambda_i^+(\eta), 
\max_{\eta \in [0,1]^d} \lambda_i^+(\eta) ]
\end{eqnarray*}}
\medskip

where the operators $T_\eta$ are defined by
\textcolor{b_bruz}{
\begin{eqnarray*}
\forall\; v \in H^1_{\#}(Y),
\quad
\ds\int_Y 
\left( \nabla T_\eta u + 2 i \pi \eta T_\eta u \right)
\cdot
\overline{\left( \nabla v + 2 i \pi \eta v \right)}
\;=\;
\\
\ds\int_\omega 
\left( \nabla u + 2 i \pi \eta u \right)
\cdot
\overline{ \left(\nabla v + 2 i \pi \eta v \right) }, \quad \eta \neq 0
\\
\forall\; v \in H^1_{\#}(Y)/\R,
\quad
\ds\int_Y 
\nabla T_0 u 
\cdot
\overline{\nabla v}
\;=\; 
\ds\int_\omega 
\nabla u
\cdot
\overline{\nabla v}, \quad \eta = 0
\end{eqnarray*}}


}
\end{frame}
%------------------------------------------------------------
\begin{frame}
\small{




\textcolor{b_bruz}{$\bullet$}
The boundary layer spectrum $\sigma_{\partial \Omega}$ is
defined as the set of $\lambda \in (0,1)$ such that 
\textcolor{b_bruz}{
\begin{eqnarray*}
\exists\; (\lambda_\e) \subset \sigma(T_\e) \quad\textrm{such that}\; \lambda_\e \to \lambda
\end{eqnarray*}}
\medskip

and for which the associated eigenvectors $u_\e \in H^1_0(\Omega)$ satisfy
\textcolor{b_bruz}{
\begin{eqnarray*}
\forall\; s > 0\quad 
\lim_{\e \to 0}
\e^{-(1 - 1/2 + s)} ||\nabla u_\e||_{L^2({\mathcal U}_\e)}
&=& \infty
\end{eqnarray*}}

where 
\begin{eqnarray*}
{\mathcal U}_\e &=& \{ x \in \Omega, d(x, \partial \Omega) < \e\}
\end{eqnarray*}

}
\end{frame}
%------------------------------------------------------------
\begin{frame}
\small{

\vspace*{10mm}

\textcolor{ox}{\bf Remarks~:} 
\begin{itemize}
\item[-]
It is more convenient to work with $T_\e$ (domains of definition easier to handle)
\medskip

\item[-]
Our work is largely inspired by the analysis of \textcolor{red}{[Allaire-Conca]}
who studied the high frequency limit of spectra of diffusion equations using
Bloch wave homogenization
\medskip

\item[-]
As $\e \to 0$, the operators $T_\e$ converge to a limiting operator $T_\infty$
defined on $H^1_0(\Omega)$ by
\begin{eqnarray*}
\forall\; v \in H^1_0(\Omega)\quad
\ds\int_\Omega \nabla T_\infty u\cdot \nabla v
&=&
|\omega|
\ds\int_\Omega \nabla u\cdot \nabla v
\end{eqnarray*}

However, the convergence is only in a weak sense, and thus does not yield any
information on $\lim_{\e \to 0} \sigma(T_\e)$
\medskip

\item[-]
To take into account the microscopic effects in the limit, we
define a 2-scale version $\tilde{T}_\e$ of $T_\e$ on the larger
space $L^2(\Omega, H^1(\omega)/\R)$, which has the same spectrum
\medskip

\item[-]
We show that the operators $\tilde{T}_\e$ converge strongly to a
limiting operator $\tilde{T}_0$, and thus
\textcolor{red}{
$\quad\quad \lim_{\e \to 0} \sigma(T_\e) \;\supset\; \sigma(\tilde{T}_0)$}
\end{itemize}


}
\end{frame}
%------------------------------------------------------------
\begin{frame}
\small{

\textcolor{ox}{\bf Key ingredient :}  
\medskip

2-scale convergence~\textcolor{red}{[Allaire, Nguetseng]}
and the associated compactness properties
\vspace*{5mm}

{\bf Theorem~:} \quad Let $u_\e$ be a bounded sequence in $L^2(\Omega)$
\medskip

1. Then there exists $u_0 \in L^2(\Omega \times L^2_\#(Y))$ such that
$u_\e$ 2-scale converges weakly to $u_0$, i.e.
\textcolor{ox}{
\begin{eqnarray*}
\forall\; \phi \in L^2(\Omega, {\mathcal C}_\#(Y)), \quad
\ds\int_\Omega u_\e(x) \phi(x, x/\e) \,dx
&\rightarrow&
\ds\int_{\Omega \times Y}
u_0(x,y) \phi(x,y)\, dxdy
\end{eqnarray*}}
\vspace*{5mm}

%%2. Assume further that 
%%\textcolor{ox}{
%%\begin{eqnarray*}
%%||u_\e||_{L^2(\Omega)} &\rightarrow& ||u_0||_{L^2(\Omega \times Y)}
%%\end{eqnarray*}}
%%Then $u_\e$ 2-scale converges strongly, i.e., for any sequence $v_\e$
%%that 2-scale converges weakly to $v_0 \in L^2(\Omega \times L^2_\#(Y))$
%%\textcolor{ox}{
%%\begin{eqnarray*}
%%\forall\; \phi \in {\mathcal C}(\overline{\Omega}, {\mathcal C}_\#(Y)), \quad
%%\ds\int_\Omega u_\e(x) v_\e(x) \phi(x,x/\e) \,dx
%%&\rightarrow&
%%\ds\int_{\Omega \times Y}
%%u_0(x,y) v_0(x,y)\phi(x,y)\, dxdy
%%\end{eqnarray*}}
%%
%%
%%}
%%\end{frame}
%------------------------------------------------------------
%%\begin{frame}
%%\small{

2. Assume that a sequence $(u_\e)$ converges weakly in $L^2$ to some
$u_0 \in H^1(\Omega)$. Then there exists $\hat{u} \in L^2(\Omega,H^1_\#(Y)/\R)$ 
such that, up to a subsequence
\medskip

\begin{itemize}
\item[-]
$u_\e$ 2-scale converges to $u$
\medskip

\item[-]
$\nabla u_\e$ 2-scale converges to $\nabla u_0(x) + \nabla_y \hat{u}(x,y)$
\end{itemize}

}
\end{frame}
%------------------------------------------------------------
\begin{frame}
\small{

\textcolor{orange}{\Large{\bf 3. Bloch wave homogenization}}
\vspace*{1cm}

Following \textcolor{red}{[Allaire-Conca] (see also [Cioranescu-Damlamian-Griso])}
we define
\medskip

\begin{itemize}
\item[-]
an extension operator $E_\e~: L^2(\Omega) \longrightarrow L^2(\Omega \times Y)$
\textcolor{ox}{
\begin{eqnarray*}
E_\e u(x,y) &=&
\left\{ \begin{array}{cl}
u(\e [x/\e] + \e y) &\textrm{if $x \in \omega_{\e,i} \subset \Omega$}
\\
\\
0 & \textrm{otherwise}
\end{array} \right.
\end{eqnarray*}}
\medskip

\item[-]
a projection operator $P_\e~: L^2(\Omega \times Y) \longrightarrow L^2(\Omega)$
\textcolor{ox}{
\begin{eqnarray*}
P_\e \phi(x) &=&
\left\{ \begin{array}{cl}
\ds\int_Y \phi( \e [x/\e] + \e z, \{x/e\} )\, dz 
&\textrm{if $x \in \omega_{\e,i} \subset \Omega$}
\\
\\
0 & \textrm{otherwise}
\end{array} \right.
\end{eqnarray*}}
\end{itemize}


}
\end{frame}
%------------------------------------------------------------
\begin{frame}
\small{

Denoting $\Omega_\e$ the union of all the cells $\omega_{\e,i}$ that are
fully contained in $\Omega$
\bigskip


\begin{itemize}
\item[$\bullet$]
$E_\e$ and $P_\e$ are bounded operators with norm 1
\medskip

\item[$\bullet$] $P_\e : L^2(\Omega \times Y) \rightarrow L^2(\Omega)$ 
and $E_\e : L^2(\Omega) \rightarrow L^2(\Omega \times Y)$
are \textcolor{b_bruz}{adjoint operators}
\medskip

\item[$\bullet$] $P_\e$ and $E_\e$ are \textcolor{b_bruz}{almost inverse} 
to one another

for $u \in L^2(\Omega)$, \quad
\textcolor{ox}{
$P_\e E_\e u(x) \;=\; 
\left\{ \begin{array}{cl}
u(x) & \textrm{if}\; x \in \Omega_\e
\\
0 & \textrm{otherwise}
\end{array} \right.$}
\medskip

for $\phi \in L^2(\Omega \times Y)$, 
\textcolor{ox}{ \quad $E_\e P_\e \phi \rightarrow \phi$
strongly in $L^2(\Omega \times Y)$}
\end{itemize}


}
\end{frame}
%------------------------------------------------------------
\begin{frame}
\small{
In our setting, we should be cautious as the definition of $T_\e$ 
involves derivatives, whereas the operators $E_\e, P_\e$ may not
define functions in $H^1$
\bigskip

\textcolor{red}{We set $\quad \tilde{T}_\e := E_\e \, T^\circ_\e \, P_\e \quad$} with
\medskip

\begin{eqnarray*}
\tilde{T}_\e~: L^2(\Omega, H^1(\omega)/\R) &\longrightarrow & L^2(\Omega, H^1(\omega)/\R)
\\
P_\e \;\downarrow && \uparrow \; E_\e
\\
{T^\circ_\e}~: H_\e := H^1(\omega_\e)/C(\omega_\e) &\longrightarrow& H_\e 
\\
\downarrow && \uparrow 
\\
T_\e~: H^1_0(\Omega) &\longrightarrow& H^1_0(\Omega)
\end{eqnarray*}
\bigskip

where \textcolor{red}{$C(\omega_\e) \;=\; 
\{ u \in H^1_0(\Omega),  \;\; u = (\textrm{const})_i \;\textrm{on}\; \omega_{\e,i} \}$}
\medskip


\textcolor{gray}{
\[ \begin{array}{l}
\phi \in L^2(\Omega \times H^1(\omega)/\R)
\;\rightarrow\;
P_\e \phi := u_\e \in H^1(\omega_\e)/C(\omega_\e)
\\
\rightarrow \; v_\e \in H^1_0(\Omega), \quad \textrm{such that} \quad
\ds\int_\Omega \nabla v_\e \cdot \nabla v \;=\; \ds\int_{\omega_\e} \nabla u_\e \cdot \nabla v
\\
\rightarrow\; \tilde{T}_\e \phi = E_\e v_\e|_{\Omega \times \omega}
\end{array}
\]}


}
\end{frame}
%------------------------------------------------------------
\begin{frame}
\small{

{\bf Proposition}
\medskip

\begin{itemize}
\item[$\bullet$]
$\tilde{T}_\e$ is self-adjoint and \quad
$\sigma(\tilde{T}_\e) \;=\; \sigma(T_\e) \setminus \{0\}$
\bigskip

\item[$\bullet$]
For any $\phi \in L^2(\Omega, H^1(\omega)/\R)$, $\tilde{T}_\e \phi$
\textcolor{b_bruz}{converges strongly} in $L^2(\Omega, H^1(\omega)/\R)$ to some $\tilde{T}_0 \phi$
\medskip

$\tilde{T}_0 \phi = Q \hat{v}$ where 
$Q~: L^2(\Omega, H^1_\#(Y)/\R) \longrightarrow L^2(\Omega, H^1(\omega)/\R)$
is the restriction operator and $\hat{v}$ is the unique solution in 
$L^2(\Omega, H^1_\#(Y)/\R)$ of
\textcolor{b_bruz}{
\begin{eqnarray*}
- \Delta_y \hat{v}(x,y) &=&
- \textrm{div}_y(1_\omega(y) \nabla_y \phi)(x,y)
\quad\quad \textrm{in}\; Y, \;a.e. \;x \in \Omega
\end{eqnarray*}}

\item[$\bullet$] It follows that
\textcolor{b_bruz}{
$\quad \lim_{\e \to 0} \sigma(T_\e) \supset \sigma(\tilde{T}_0)$}
\bigskip

\item[$\bullet$] Actually, $\sigma(\tilde{T}_0) = \sigma(T_0) \setminus\{0\}$, 
where $T_0~: H^1_\#(Y)/\R \longrightarrow H^1_\#(Y/\R)$ is defined by
\textcolor{b_bruz}{
\begin{eqnarray*}
\forall\; v \in H^1_\#(Y), \quad
\ds\int_Y \nabla T_0 u \cdot \nabla v
&=&
\ds\int_\omega \nabla u \cdot \nabla v
\end{eqnarray*}}
\end{itemize}
The values in $\sigma(T_0)$ can be interpreted 
as eigenvalues of single-cell resonant modes
}
\end{frame}
%------------------------------------------------------------
\begin{frame}
\small{

This follows from the compactness induced by 2-scale convergence~:
\medskip

\textcolor{b_bruz}{
\[ \begin{array}{l}
\phi \in L^2(\Omega \times H^1(\omega)/\R)
\;\rightarrow\;
P_\e \phi := u_\e \in H^1(\omega_\e)/C(\omega_\e)
\\
\rightarrow \; v_\e \in H^1_0(\Omega), \quad \textrm{such that} \quad
\ds\int_\Omega \nabla v_\e \cdot \nabla v \;=\; \ds\int_{\omega_\e} \nabla u_\e \cdot \nabla v
\\
\rightarrow\; \tilde{T}_\e \phi = E_\e v_\e|_{\Omega \times \omega}
\end{array}
\]}

then 

\[
\left\{
\begin{array}{lcll}
\e v_\e & \rightharpoonup & v_0 &\quad \textrm{weakly in}\; H^1_0(\Omega)
\\
\e E_\e(\nabla v_\e) &\rightharpoonup & \nabla v_0 + \nabla_y \hat{v}
&\quad \textrm{weakly in}\; L^2(\Omega \times Y)
\end{array} \right. 
\]
\begin{eqnarray*}
\ds\int_{\Omega \times Y}
(\nabla v_0 + \nabla_y \hat{v}) \cdot (\nabla \phi + \nabla_y \psi)
&=&
\ds\int_{\Omega \times \omega}
\nabla_y \phi(x,y) \cdot (\nabla \phi + \nabla_y \psi)
\end{eqnarray*}

}
\end{frame}
%------------------------------------------------------------
\begin{frame}
\small{

\textcolor{ox}{\bf Collective resonances of the inclusions}
\vspace*{5mm}

The rescaling procedure can also be performed on a pack of cells
(i.e. over $K^d$ copies of the unit cell $Y$)
\medskip

\begin{itemize}
\item[-] define corresponding projection and extension operators
$E_\e^K, P_\e^K$
\medskip

\item[-] define $\tilde{T}_\e^K$ 
\medskip

\item[-]
show that $\tilde{T}_\e^K$ converges strongly to a limiting operator $\tilde{T}_0^K$
\medskip

\item[-]
whose spectrum coincides with that of 
$T_0^K~: H^1_\#(KY)/\R \longrightarrow H^1_\#(KY)/\R$
defined by
\textcolor{ox}{
\begin{eqnarray*}
\forall\; v \in H^1_\#(KY), \quad
\ds\int_{KY} \nabla T_0 u \cdot \nabla v
&=&
\ds\int_{\omega^K} \nabla u \cdot \nabla v
\end{eqnarray*}}
\medskip

\item[-] and in fact
\textcolor{ox}{
$\quad \sigma(T_0^K) \;=\;\ \cup_{0 \leq j \leq K-1} \sigma(T_{\eta})
\quad\quad \eta = j/K $}
\end{itemize}

}
\end{frame}
%------------------------------------------------------------
\begin{frame}
\small{

\textcolor{orange}{\bf{\Large 4. Homogenization with NIM's }}
\bigskip


Let $f \in L^2(\Omega)$ and consider $u_\e \in H^1_0(\Omega)$ solution to
\textcolor{red}{
\begin{eqnarray*}
(P_\e) \quad\quad
- \textrm{div}(A_\e(x) \nabla u_\e(x)) &=& f 
\quad\quad \textrm{in}\; \Omega
\end{eqnarray*}}

where $A_\e(x) \;=\; \left\{ \begin{array}{ll}
\textcolor{red}{a > 0} & x \in \omega_\e
\\
1 & \textrm{otherwise}
\end{array} \right.$
\medskip

Then $u_\e \rightharpoonup u_*$ weakly in $H^1_0(\Omega)$, with
\textcolor{red}{
\begin{eqnarray*}
(P_*) \quad\quad
- \textrm{div}(A_* \nabla u_*(x)) &=& f 
\quad\quad \textrm{in}\; \Omega
\end{eqnarray*}}
\medskip

$A_*$ is a (constant) matrix, whose entries
are given in terms of the solutions to the cell problems~: 
find $\chi_j \in H^1_\#(Y)/\R$ such that
\textcolor{red}{
\begin{eqnarray*}
- \textrm{div}(A(y)\nabla(\chi_j(x) + y_j)) &=& 0 
\quad\quad \textrm{in}\; Y
\end{eqnarray*}}
\medskip

}
\end{frame}
%------------------------------------------------------------
\begin{frame}
\small{

What happens in the more general case when $a \in {\mathbb C}$ ?
\medskip

\textcolor{red}{[Bouchitt\'e-Bourel-Feldbacq, Hoai-Minh Nguyen,
Bunoiu-Ramdani,...]}
\bigskip

Note that if $\lambda = 1/(1-a)$ is not in the spectrum of $T_0~: H^1_\#(y) \rightarrow H^1_\#(Y)$
\begin{eqnarray*}
\forall\; v \in H^1_\#(Y) \quad
\ds\int_Y \nabla T_0u \cdot \nabla v
&=& \ds\int_\omega \nabla u \cdot \nabla v
\end{eqnarray*}

the homogenized tensor is formally well defined

}
\end{frame}
%------------------------------------------------------------
\begin{frame}
\small{

\textcolor{ox}{\bf Prop.}
\medskip

Let $f \in H^{-1}(\Omega)$. Assume that 
$\lambda = 1/(1-a) \notin \sigma(T_0)$ so that $A^*$ is well defined
\medskip

\begin{itemize}
\item[-] 
If $u_\e$ is a sequence of solutions to $(P_\e)$
such that \textcolor{ox}{$u_\e \to u$} weakly in $H^1$, then 
$u$ is a solution to $(P_*)$
\medskip

\item[-]
If $u$ is a solution to $(P_*)$ (if any), then there exists a sequence
$(f_\e) \subset H^{-1}(\Omega), f_\e \to f$ such that 
the solutions $u_\e \in H^1_0(\Omega)$ to
\textcolor{ox}{
\begin{eqnarray*}
- \textrm{div}(A_\e \nabla u_\e) &=& f_\e \quad \textrm{in}\; \Omega
\end{eqnarray*}}
satisfy \textcolor{ox}{$u_\e \rightharpoonup u$ weakly in $H^1(\Omega)$}
\end{itemize}
\bigskip

\textcolor{gray}{In particular, homogenization cannot discriminate
among solutions to the homogenized equation, if they are not unique}

}
\end{frame}
%------------------------------------------------------------
\begin{frame}
\small{

We can then relate (partially) the limiting spectrum with 
the homogenization tensor
\bigskip

\textcolor{ox}{\bf Prop.}
\medskip

Let $a \in {\mathbb C} \setminus \sigma(T_0)$ and let $A_*$ denote the
associated homogenized matrix
\medskip

Assume that there exists $f \in H^{-1}(\Omega)$ such that the PDE
\textcolor{ox}{
\begin{eqnarray*}
- \textrm{div}(A^* \nabla u) &=& f \quad\quad \textrm{in}\; \Omega
\end{eqnarray*}}
does not have a solution in $H^1_0(\Omega)$
\medskip

Then \textcolor{ox}{$\quad 1/(1-a) \in \lim_{\e \to 0} \sigma(T_\e)$}
\bigskip

\textcolor{b_bruz}{The converse is false: the case of rank-one laminates
shows that the above system can be well-posed when a is in
the limiting spectrum}

}
\end{frame}
%------------------------------------------------------------
\begin{frame}
\small{

\textcolor{ox}{\bf{\Large High contrast ($a \to \pm \infty$ or $a \to 0$)}}
\vspace*{10mm}

Recall that we assumed that $\omega \subset\subset Y$
\medskip

{\bf Prop.} 

There exists $-\infty < c < C < 0$ such that
if $\quad -\infty < a < c \quad$ or if $\quad C < a < 0$
\medskip

\begin{itemize}
\item[\textcolor{b_bruz}{$\bullet$}]
($P_\e$) is well posed and its solution $u_\e$ depends continuously on $f$
\medskip

\item[\textcolor{b_bruz}{$\bullet$}]
The homogenized tensor $A^*$ is elliptic (uniform bounds
wrt $a$)
\medskip

In particular the homogenized problem $(P_*)$ is well-posed.
\end{itemize}

}
\end{frame}
%------------------------------------------------------------
\begin{frame}
\small{

\textcolor{b_bruz}{\bf{\Large 5. Conclusion/perspectives}}
\vspace*{10mm}

\begin{itemize}
\item[-]
Does the Bloch spectrum really play a role wrt resonance ?
\medskip

\item[-]
How to better characterize the boundary spectrum
\medskip

\item[-]
What if the inclusions are not smooth ?
\medskip

\item[-]
The hypothesis $\omega \subset\subset Y$ plays an important 
role. Laminates provide counter-examples to some of the properties
we derived
\medskip

\item[-]
Is it possible to construct hyperbolic media under the
hypothesis that $\omega \subset\subset Y$ ?
\end{itemize}

}
\end{frame}
%------------------------------------------------------------
\begin{frame}
\small{

\begin{figure}[hbt]
\hfill
\includegraphics[angle=0,width=30mm]{../Incl/incl_nontouch_3.jpg}
\end{figure}
\vspace*{-2cm}


}
\end{frame}


%%--------------------------------------------------------------------------------
\end{document}
%%--------------------------------------------------------------------------------





